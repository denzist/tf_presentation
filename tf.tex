\documentclass[9pt]{beamer}

% \usepackage[T1]{fontenc}
\usepackage[utf8]{inputenc}
\usepackage[russian]{babel}
\usepackage{graphicx}
\usepackage{hyperref}
\graphicspath{ {images/} }

\usetheme{Goettingen}


\title{Обзор библиотеки ros tf}

% A subtitle is optional and this may be deleted
% \subtitle{Optional Subtitle}

% \author{Денис Шепелев\inst{1}}
\author{Денис Шепелев}

% - Give the names in the same order as the appear in the paper.
% - Use the \inst{?} command only if the authors have different
%   affiliation.

\institute[] % (optional, but mostly needed)
{
  % \inst{1}%
  % 073а\\
}
% - Use the \inst command only if there are several affiliations.
% - Keep it simple, no one is interested in your street address.

\date{ }
% \date{}

\begin{document}

\begin{frame}
  \titlepage
\end{frame}

\section{Как работает tf}

\subsection{Задачи и структура tf}

\begin{frame}{Задачи и структура tf}
  Основные задачи tf
  \begin{itemize}
    \item
    {
      обеспечить надежную и эффективную работу в распределенной системе
    }
    \item
    {
      простоту изменения и получения данных
    }
  \end{itemize}

\end{frame}

\begin{frame}{Задачи и структура tf}
  \begin{itemize}
    \item
    {
      Основная структура - граф
    }
    \item
    {
      Вершина - система координат (frame)
    }
    \item
    {
      Ребро - преобразование (transform) между соединенными этим ребром фреймами
    }
  \end{itemize}

  Никаких явных ограничений на граф нет, однако рекомендуется
  \begin{itemize}
    \item
    {
      чтобы граф был связный
    }
    \item
    {
      и ацикличный
    }
  \end{itemize}

\end{frame}

\begin{frame}{Задачи и структура tf}
Типичный запрос в tf выглядит следующим образом:
\begin{itemize}
    \item
    {
      Получить преобразование transform между parent и child фреймами в момент времени t
    }
  \end{itemize}

Поэтому для каждого ребра в хронологическом порядке хранится история его изменений в течение некоторого промежутка времени. 

\end{frame}

\subsection{Listener и Broadcaster}

\begin{frame}{Listener и Broadcaster}
Основные компоненты для работы с tf
\begin{itemize}
    \item
    {
      Broadcaster - периодически отправлять данные в tf
    }
    \item
    {
      Listener - собирает данные, которые приходят в tf, и работает с запросами
    }
  \end{itemize}

\end{frame}

\begin{frame}{Listener и Broadcaster}
Следует отметить, что Listener разрешает запросы используя интерполяцию (SLERP = spherical linear interpolation). Поэтому
\begin{itemize}
    \item
    {
      необходимо обеспечить достаточную быструю частоту публикаций данных
    }
    \item
    {
      с учетом пропускной способности сети
    }
  \end{itemize}

Благодаря интерполяции система может работать асинхронно, вдобавок повышая устойчивость в случае потери данных.

\end{frame}

\begin{frame}{Listener и Broadcaster}
Механизм разрешения запроса
\begin{itemize}
    \item
    {
      Listener гуляет по графу, формируя остовное дерево содержащее child и parent
    }
    \item
    {
      Если parent не найдено, будет получено соответствующее исключение
    }
    \item
    {
      Если остовное дерево сформировано, то вдоль ребер, лежащих на пути между child и parent, ищется преобразование между целевыми фреймами
    }
    \item
    {
      На каждом ребре ищется интерполяция в момент времени, заданный в запросе, используя 2 ближайших по времени преобразования. Если все они существуют, то между фреймами $a$ и $c$, между которыми в графе лежит фрейм $b$, преобразование может быть найдено по формуле
      $$T^c_a = T^b_a \ast T^c_b $$ 
    }
  \end{itemize}
\end{frame}

\section{Работа с tf}

\subsection{Data}

\begin{frame}{Data}
Основные структуры, которыми оперирует tf
  \begin{itemize}
    \item
    {
      tf::Quaternion
    }
    \item
    {
      tf::Vector3
    }
    \item
    {
      tf::Point
    }
    \item
    {
      tf::Pose
    }
    \item
    {
      tf::Transform
    }
    \item
    {
     tf::Stamped <T>
    }
    \item
    {
     tf::StampedTransform
    }
  \end{itemize}

Стоит отдельно отметить полезную функцию 

tf::Quaternion createQuaternionFromRPY(double roll,double pitch,double yaw)
\end{frame}

\subsection{Broadcaster}


\begin{frame}{Broadcaster}
Для того чтобы отправить преобразование достаточно сделать следующее
  \begin{itemize}
    \item
    {
      tf::TransformBroadcaster()
    }
    \item
    {
      void sendTransform(const StampedTransform \& transform) - отправляет преобразование
    }
  \end{itemize}
\end{frame}

\subsection{Listener}

\begin{frame}{Listener}
И наконец, для того чтобы можно было посылать запросы системе используется 
  \begin{itemize}
    \item
    {
      tf::TransformListener
    }
    \item
    {
      bool canTransform(...) - проверяет, может ли преобразование быть найдено
    }
    \item
    {
      bool waitForTransform(...) - блокирует исполнение, на некоторый промежуток времени, пока преобразование не будет найдено, или не закончится время 
    }
    \item
    {
      void lookupTransform(...) - находит преобразование, в случае неудачи выбрасывает исключение
    }
  \end{itemize}
\end{frame}

\begin{frame}{Listener}
Также Listener предоставляет возможность преобразовывать данные между фреймами, такие как Quaternion, Vector3, Pose и т.д.
  \begin{itemize}
    \item
    {
      void tf::TransformListener::transformDATATYPE(...)
    }
  \end{itemize}
\end{frame}

\begin{frame}{Исключения}
Некоторые методы выбрасывают исключения, которые наследуются от tf::TransformException 
  \begin{itemize}
    \item
    {
      tf::ConnectivityException - если 2 фрейма не принадлежат одному связанному графу
    }
    \item
    {
      tf::ExtrapolationException - если между 2 фреймами одно и более устаревших связей
    }
    \item
    {
      tf::InvalidArgument - при неправильных аргументах
    }
    \item
    {
      tf::LookupException - если в запросе один из фреймов не существует 
    }
  \end{itemize}
\end{frame}

\section{Инструменты командной строки}
\subsection{Инструменты командной строки}

\begin{frame}{Инструменты командной строки}
Основные инструменты командной строки 
  \begin{itemize}
    \item
    {
      view\_frames - визуализирует в pdf структуру графа tf
    }
    \item
    {
      tf\_monitor - мониторит преобразования между фреймами (всеми, либо 2 заданными) 
    }
    \item
    {
      tf\_echo - выводит преобразование между 2 заданными фреймами
    }
    \item
    {
      roswtf - клевая тулза которая позволяет найти много разных проблем в ros, в том числе и tf, например, выведет предупреждение о том, что граф не связный
    }
    \item
    {
      static\_transform\_publisher - позволяет публиковать статическое преобразование
    }
  \end{itemize}
\end{frame}

\section{Возможные проблемы}
\subsection{Возможные проблемы}

\begin{frame}{Возможные проблемы}
Возможные проблемы при работе с tf
  \begin{itemize}
    \item
    {
      \textcolor{red}{Проблема} - Некоторые ноды ros могут публиковать в tf противоречивую информацию (циклы в графе, несвязный граф).

      \textcolor{green}{Решение} - Искать баг с помощью инструментов командной строки, в самом начале разработки внимательно следить и регулировать кто что публикует.
    }
    \item
    {
      \textcolor{red}{Проблема} - tf ругается на устаревшие данные.

      \textcolor{green}{Решение} - Возможно вы запустили rosbag play и забыли прописать --clock или /use\_sim\_time = true в rosparam.
    }
    \item
    {
      \textcolor{red}{Проблема} - Сообщения приходят быстрее чем, обновляется информация в tf. 

      \textcolor{green}{Решение} - Использовать waitForTransform или tf::MessageFilter
    }
  \end{itemize}
\end{frame}


% \begin{frame}{2D карта}
% \begin{columns}
%   \begin{column}{0.50\textwidth}
%     \begin{itemize}
%       \item
%       {
%         Карта - обычное изображение
%       }
%       \item
%       {
%         Каждый пиксель - некоторая область пространства
%       }
%       \item
%       {
%         Белый пиксель - свободная для движения область
%       }
%       \item
%       {
%         Черный - чем-то занятая облать
%       }
%       \item
%       {
%         Серый - неизвестная область
%       }
%     \end{itemize}
%   \end{column}
%   % \begin{column}{0.50\textwidth}
%   %   \begin{figure}[h]
%   %     \centering
%   %     \includegraphics[width=0.8\textwidth]{grid_map.png}
%   %   \end{figure}
%   % \end{column}
% \end{columns}
% \end{frame}



% \begin{frame}{Name}
% \begin{itemize}
%   \item
%   {
      
%   }
% \end{itemize}
% \end{frame}


% All of the following is optional and typically not needed. 
\appendix
\section<presentation>*{\appendixname}
\subsection<presentation>*{Источники}

\begin{frame}[allowframebreaks]
  \frametitle<presentation>{Источники}
    
  \begin{thebibliography}{10}
    
  \beamertemplatearticlebibitems
  % Followed by interesting articles. Keep the list short. 

  \bibitem{tf}
    \newblock {TF}
    \newblock {\em http://wiki.ros.org/tf}


  \end{thebibliography}
\end{frame}

\end{document}


